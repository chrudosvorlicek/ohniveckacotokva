\documentclass[12pt, a4paper]{article}

\usepackage[czech]{babel}
\usepackage[utf8]{inputenc}
\usepackage{graphicx}

\usepackage{multicol}
\setlength{\columnsep}{1cm}
\setlength{\parindent}{0em}

\usepackage{geometry}
\geometry{margin=1cm}

\begin{document}
\textbf{\huge Konfliktní typy} \\
\begin{tabular}{ | p{0.2\textwidth}| p{0.37\textwidth} | p{0.37\textwidth}|}
\hline
Typ &  Charakteristika & Jak na ně \\ \hline
\hline
ÚZKOSTNÝ &
Schoulené držení těla, těkavý pohled, klopení zraku, nejistá gesta, opakování dotazů. Bojí se učinit rozhodnutí, pochybnosti, výčitky, sebeobviňování. Nevěří si, volí únikové cesty.  Odvolává se na autority.
& 
Signalizuj dopředu dostatek času. Uvažuj za něho nahlas, klaď nevyslovené otázky, vyslov za něho pochyby a odpovídej pak na ně. Nečiň přímé rozhodnutí za něj, ale přispěj k jeho řešení obecným shrnutím závěrů, naznač možné alternativy řešení.
\\ \hline
ÚZKOSTNĚ AGRESIVNÍ
&
Energicky pevný úchop, sevřené rty, strohý řečový projev, nervozita v hlase. Pocit křivdy spojen s rychlými obranářskými reakcemi. Odmítá projevy pomoci a péče. Zraňující výroky, vedené na city druhých.
&
Vyčkej, až se na nás obrátí, ne pozice tváří v tvář. Hovoř pomalu, klidně, podej vyčerpávající informace. Vyvaruj se otázek buď a nebo, netlač ho k rychlému rozhodnutí
\\ \hline
NARCISTNÍ
&
Časté upravování vlastního zevnějšku, okázalé vystupování, přezíravá mimika, bohatá gestikulace. Rád se poslouchá. V situaci ponížení reaguje okázale rezervovaným vyčkáváním, ironií, sarkasmem, ostrou kritikou.
V konfliktu se prezentuje jako mocensky silný, člověk vlivných konexí a netušených možností.
&
Jdi vstříc jako první, ne blíže než jeden metr. Nelze-li se mu věnovat ihned, signalizuj, že o něm víš a vyžádej si strpení. Dávej mu najevo, že je středem tvé pozornosti. Nepoučuj,nebuď podbízivý, ponížený. Jednej s ním jako se zasvěceným partnerem. 
\\ \hline
BEZOHLEDNÝ -- AGRESIVNÍ
&
Nechápavý/nepřátelský výraz obličeje, pomalé myšlení, podezíravost. Řečový projev hlučný, nespisovný, vulgární výrazy. Netolerantní prosazování svých zájmů, výbušný. Nepřátelský postoj vůči obecně uznávaným hodnotám, konvencím a zvyklostem.
&
Nechoď blíže než na vzdálenost 1m, ale necouvej, pohled z očí do očí. Neklop ani nezvedej oči. Mluv klidně, věcně, pomalu, ale nevzrušeně bez afektu. Nejednej zmateně ani zakřiknutě. Nepodléhej panice! Ne přehnanou kulturu projevu! Projev pochopení a uznání, aniž bys ustoupil ze svých pozic. Ale na druhou stranu nabídni možnost důstojného ústupu.
\\ \hline
PEDANTNÍ -- PEDANTERNÍ
&
Pevné držení těla, střízlivé oblečení, disciplinovaná gesta, ritualizovaný životní styl, pořádkumilovnost, obdiv k předpisům, nařízením. Přísně logické myšlení. Zdrženlivý vůči novým věcem. Pokud se dostane do situace, která je v rozporu s jeho vnitřním řádem, reaguje zlobně.
&
Vzdálenost kolem 1m, jednat klidně, věcně. Nezahlcovat informacemi! Dej najevo příbuznost názorů, respekt k tradici. Projev pochopení. Lze se dovolávat chronologie postupů. Pokud dojde ke konfliktu, projev úctu k jeho osobě, uznej právo jeho svobodného rozhodování se.
\\ \hline
NEPŘÍSTUPNÝ
&
Nezájem, přehlížení, pohrdání, případně vtipkování. Nemluvnost, dívá se jinam, zkřížené ruce na prsou.
&
Nepřistupuj k němu jako první, nechej ho určit osobní vzdálenost. Mluv stručně a věcně. Nenech se vyvést z míry jeho mlčení. Nereaguj na narážky a dvojsmysly. Neuspěchej jednání.
\\ \hline
HISTORION
&
Rychlá chůze, prudká gesta. Rád na sebe poutá pozornost druhých, do hry vtahuje ostatní. Řečový projev velmi emocionální, může postrádat logiku. Při konfliktu může dojít k projevům agrese.
&
Zbav histriona publika. Jednej v sedě, omez tak možnost pohybu a gestikulace. Buď vstřícný, ale vracej vždy jednání k předmětu věci. Buď rozhodný, rázný. Nenech se vytočit, zůstaň přátelský.
\\ \hline
\end{tabular}

\newpage
\textbf{\huge Konfliktní typy} \\
\begin{tabular}{ | p{0.2\textwidth}| p{0.37\textwidth} | p{0.37\textwidth}|}
\hline
Typ &  Charakteristika & Jak na ně \\ \hline
\hline
ÚZKOSTNÝ &
Schoulené držení těla, těkavý pohled, klopení zraku, nejistá gesta, opakování dotazů. Bojí se učinit rozhodnutí, pochybnosti, výčitky, sebeobviňování. Nevěří si, volí únikové cesty.  Odvolává se na autority.
& 
Signalizuj dopředu dostatek času. Uvažuj za něho nahlas, klaď nevyslovené otázky, vyslov za něho pochyby a odpovídej pak na ně. Nečiň přímé rozhodnutí za něj, ale přispěj k jeho řešení obecným shrnutím závěrů, naznač možné alternativy řešení.
\\ \hline
ÚZKOSTNĚ AGRESIVNÍ
&
Energicky pevný úchop, sevřené rty, strohý řečový projev, nervozita v hlase. Pocit křivdy spojen s rychlými obranářskými reakcemi. Odmítá projevy pomoci a péče. Zraňující výroky, vedené na city druhých.
&
Vyčkej, až se na nás obrátí, ne pozice tváří v tvář. Hovoř pomalu, klidně, podej vyčerpávající informace. Vyvaruj se otázek buď a nebo, netlač ho k rychlému rozhodnutí
\\ \hline
NARCISTNÍ
&
Časté upravování vlastního zevnějšku, okázalé vystupování, přezíravá mimika, bohatá gestikulace. Rád se poslouchá. V situaci ponížení reaguje okázale rezervovaným vyčkáváním, ironií, sarkasmem, ostrou kritikou.
V konfliktu se prezentuje jako mocensky silný, člověk vlivných konexí a netušených možností.
&
Jdi vstříc jako první, ne blíže než jeden metr. Nelze-li se mu věnovat ihned, signalizuj, že o něm víš a vyžádej si strpení. Dávej mu najevo, že je středem tvé pozornosti. Nepoučuj,nebuď podbízivý, ponížený. Jednej s ním jako se zasvěceným partnerem. 
\\ \hline
BEZOHLEDNÝ -- AGRESIVNÍ
&
Nechápavý/nepřátelský výraz obličeje, pomalé myšlení, podezíravost. Řečový projev hlučný, nespisovný, vulgární výrazy. Netolerantní prosazování svých zájmů, výbušný. Nepřátelský postoj vůči obecně uznávaným hodnotám, konvencím a zvyklostem.
&
Nechoď blíže než na vzdálenost 1m, ale necouvej, pohled z očí do očí. Neklop ani nezvedej oči. Mluv klidně, věcně, pomalu, ale nevzrušeně bez afektu. Nejednej zmateně ani zakřiknutě. Nepodléhej panice! Ne přehnanou kulturu projevu! Projev pochopení a uznání, aniž bys ustoupil ze svých pozic. Ale na druhou stranu nabídni možnost důstojného ústupu.
\\ \hline
PEDANTNÍ -- PEDANTERNÍ
&
Pevné držení těla, střízlivé oblečení, disciplinovaná gesta, ritualizovaný životní styl, pořádkumilovnost, obdiv k předpisům, nařízením. Přísně logické myšlení. Zdrženlivý vůči novým věcem. Pokud se dostane do situace, která je v rozporu s jeho vnitřním řádem, reaguje zlobně.
&
Vzdálenost kolem 1m, jednat klidně, věcně. Nezahlcovat informacemi! Dej najevo příbuznost názorů, respekt k tradici. Projev pochopení. Lze se dovolávat chronologie postupů. Pokud dojde ke konfliktu, projev úctu k jeho osobě, uznej právo jeho svobodného rozhodování se.
\\ \hline
NEPŘÍSTUPNÝ
&
Nezájem, přehlížení, pohrdání, případně vtipkování. Nemluvnost, dívá se jinam, zkřížené ruce na prsou.
&
Nepřistupuj k němu jako první, nechej ho určit osobní vzdálenost. Mluv stručně a věcně. Nenech se vyvést z míry jeho mlčení. Nereaguj na narážky a dvojsmysly. Neuspěchej jednání.
\\ \hline
HISTORION
&
Rychlá chůze, prudká gesta. Rád na sebe poutá pozornost druhých, do hry vtahuje ostatní. Řečový projev velmi emocionální, může postrádat logiku. Při konfliktu může dojít k projevům agrese.
&
Zbav histriona publika. Jednej v sedě, omez tak možnost pohybu a gestikulace. Buď vstřícný, ale vracej vždy jednání k předmětu věci. Buď rozhodný, rázný. Nenech se vytočit, zůstaň přátelský.
\\ \hline
\end{tabular}

\end{document}